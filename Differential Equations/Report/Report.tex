\documentclass[14pt,a4paper]{extarticle}

\usepackage[utf8]{inputenc}
\usepackage[T2A]{fontenc}
\usepackage{amssymb,amsmath,mathrsfs,amsthm}
\usepackage[russian]{babel}
\usepackage{graphicx}
\usepackage[footnotesize]{caption2}
\usepackage{indentfirst}
\usepackage{multicol}
\usepackage{listings}
\usepackage{float}
\usepackage{url}
\usepackage{hyperref}
\usepackage{enumitem}

%\usepackage[ruled,section]{algorithm}
%\usepackage[noend]{algorithmic}
%\usepackage[all]{xy}
\usepackage{booktabs}
\usepackage{graphicx}
\usepackage[table,xcdraw]{xcolor}
\usepackage{tcolorbox}

%Библиотека для блок-схем
\usepackage{tikz}
\usetikzlibrary{shapes,arrows}

% Параметры страницы
\textheight=24cm
\textwidth=16cm
\oddsidemargin=5mm
\evensidemargin=-5mm
\marginparwidth=36pt
\topmargin=-1cm
\footnotesep=3ex
%\flushbottom
\raggedbottom
\tolerance 3000
% подавить эффект "висячих стpок"
\clubpenalty=10000
\widowpenalty=10000
%\renewcommand{\baselinestretch}{1.1}
\renewcommand{\baselinestretch}{1.5} %для печати с большим интервалом

\newcommand{\angstrom}{\mbox{\normalfont\AA}}

\newtheorem{definition}{Определение} % задаём выводимое слово (для определений)
\newtheorem{example}{Замечание} % задаём выводимое слово (для определений)
\newtheorem{theorem}{Теорема} % задаём выводимое слово (для определений)
\newtheorem{construction}{Конструкция} % задаём выводимое слово (для определений)

\DeclareMathOperator*{\sgn}{sgn}
\DeclareMathOperator*{\var}{var}
\DeclareMathOperator*{\cov}{cov}
\DeclareMathOperator*{\law}{Law}

\newcommand{\1}{\mathbbm{1}}
\newcommand{\R}{\mathbb{R}}
\newcommand{\N}{\mathbb{N}}
\newcommand{\Z}{\mathbb{Z}}
\renewcommand{\P}{\mathbb{P}}
\newcommand{\E}{\mathbb{E}}

\newcommand{\independent}{\perp\!\!\!\!\perp}

\newcommand\cA{{\cal A}}
\newcommand\cE{{\cal E}}
\newcommand\cC{{\cal C}}
\newcommand\cF{{\cal F}}
\newcommand\cG{{\cal G}}
\newcommand\cK{{\cal K}}
\newcommand\cL{{\cal L}}
\newcommand\cB{{\cal B}}
\newcommand\cN{{\cal N}}
\newcommand\cM{{\cal M}}
\newcommand\cX{{\cal X}}
\newcommand\cD{{\cal D}}
\newcommand\cR{{\cal R}}
\newcommand\cP{{\cal P}}
\newcommand\cQ{{\cal Q}}
\newcommand\cS{{\cal S}}
\newcommand\cT{{\cal T}}
\newcommand\cV{{\cal V}}
\newcommand\cZ{{\cal Z}}

\newcommand{\textProposition}    {Предложение}
\newcommand{\textTask}    {Задача}

\begin{document}

\begin{center}

    {Всеволод Заостровский, 409 группа}\\
    {\bfseries Отчёт по задаче ''Итерационные методы решения систем линейных уравнений''.\\}
    \vspace{1cm}

\end{center}

\textbf{Постановка задачи.} Для построения приближенного решения задачи
$$
y^{\prime}(x)+A y(x)=0, \quad y(0)=1, \quad x \in[0,1]
$$

с известным точным решением $y(x)=e^{-A x}$ рассматриваются следующие схемы: \\
1) $\frac{y_{k+1}-y_k}{h}+A y_k=0, y_0=1$. \\
2) $\frac{y_{k+1}-y_k}{h}+A y_{k+1}=0, y_0=1$.\\
3) $\frac{y_{k+1}-y_k}{h}+A \frac{y_{k+1}+y_k}{2}=0, y_0=1$. \\
4) $\frac{y_{k+1}-y_{k-1}}{2 h}+A y_k=0, y_0=1, y_1=1-A h$.\\
5) $\frac{1.5 y_k-2 y_{k-1}+0.5 y_{k-2}}{h}+A y_k=0, y_0=1, y_1=1-A h$.\\
6) $\frac{-0.5 y_{k+2}+2 y_{k+1}-1.5 y_k}{h}+A y_k=0, y_0=1, y_1=1-A h$.\\

Найти порядок аппроксимации, исследовать $\alpha$-устойчивость предложенных схем. Реализовать указанные схемы и заполнить таблицу.

\textbf{Решение.} 
Реализацию кода см. \href{https://github.com/VsevolodZaostrovsky/NumericalMethods/tree/main/Differential%20Equations/Code/src}{тут}. 
Для реализации численного решения необходимо в каждом случае выразить последний $y_k$ через предыдущие, заодно проверим 
$\alpha$-устойчивость: \\
1) $\frac{y_{k+1}-y_k}{h}+A y_k=0 \\ 
y_{k+1} = y_k (1 - Ah)$. \\
$\lambda = 1 - A h < 1$. \\
2) $\frac{y_{k+1}-y_k}{h}+A y_{k+1}=0 \\ 
y_{k+1} = \frac{y_k}{1 + Ah} $.\\
$\lambda = \frac{1}{1 + Ah} < 1$. \\
3) $\frac{y_{k+1}-y_k}{h}+A \frac{y_{k+1}+y_k}{2}=0 \\ 
y_{k+1} = \frac{y_k (2 - A h)}{2 + Ah} $. \\
$\lambda = \frac{2 - A h}{2+ A h} < 1$. \\
4) $\frac{y_{k+1}-y_{k-1}}{2 h}+A y_k=0 \\
 y_{k+2} = y_{k} - 2 A h y_{k+1} $.\\
 $\lambda^2 + 2 A h \lambda - 1 = 0$. \\
 $\lambda_{+, -} = A h \pm \sqrt{A^2 h^2 + 1}$. \\
 $|\lambda_{+}| > 1 $. \\
5) $\frac{1.5 y_k-2 y_{k-1}+0.5 y_{k-2}}{h}+A y_k=0$ \\ 
$y_{k+2} = \frac{2 y_{k+1} - 0.5 y_{k}}{A h + 1.5} $.\\
$\lambda^2 - \frac{2}{1.5 + Ah} \lambda + \frac{0.5}{1.5 + Ah} = 0$.\\
$2 \lambda_{+, -} = \frac{2}{1.5 + Ah} \pm \sqrt{\frac{4}{(1.5 + Ah)^2} - \frac{2}{1.5 + Ah} }$. \\
$\alpha$-устойчивость есть не всегда. \\
6) $\frac{-0.5 y_{k+2}+2 y_{k+1}-1.5 y_k}{h}+A y_k=0$ \\ 
$y_{k+2} = (2 A h - 3) y_k + 4 y_{k+1}$.\\
$\lambda^2  - 4 \lambda - (2 A h - 3) = 0$.\\
$\lambda_{+, -} = 2 \pm \sqrt{1 - 2 A h}$.\\
$|\lambda_{+}| > 1 $. \\

\begin{table}[h!] 
    \begin{center} 
    \begin{tabular}{|c|c|c|c|c|c|c|} 
    \hline 
    Номер  & $E_1$ & $E_2$ & $E_3$ & $E_6$ & $m$ & $A$ \\ \hline
     $1$ &  $0.019149$ & $0.001847$ & $0.000184$ & $0.000000$ & $1$ & $1$ \\ \hline 
    
     $1$ &  $0.367879$ & $0.019201$ & $0.001847$ & $0.000002$ & $1$ & $10$ \\ \hline 
    
     $1$ &  $\infty$ & $\infty$ & $0.367879$ & $0.000184$ & $1$ & $1000$ \\ \hline 
    
     $2$ &  $0.017528$ & $0.001832$ & $0.000184$ & $0.000000$ & $1$ & $1$ \\ \hline 
    
     $2$ &  $0.132121$ & $0.017664$ & $0.001832$ & $0.000002$ & $1$ & $10$ \\ \hline 
    
     $2$ &  $0.009901$ & $0.090864$ & $0.132121$ & $0.000184$ & $1$ & $1000$ \\ \hline 
    
     $3$ &  $0.000305$ & $0.000003$ & $0.000000$ & $0.000000$ & $2$ & $1$ \\ \hline 
    
     $3$ &  $0.034546$ & $0.000307$ & $0.000003$ & $0.000000$ & $2$ & $10$ \\ \hline 
    
     $3$ &  $0.960784$ & $0.666712$ & $0.034546$ & $0.000000$ & $2$ & $1000$ \\ \hline 
    
     $4$ &  $1.573393$ & $1.546693$ & $1.543448$ & $1.543081$ & $0$ & $1$ \\ \hline 
    
     $4$ &  $\infty$ & $\infty$ & $\infty$ & $\infty$ & $0$ & $10$ \\ \hline 
    
     $4$ &  $\infty$ & $\infty$ & $\infty$ & $\infty$ & $0$ & $1000$ \\ \hline 
    
     $5$ &  $0.431269$ & $0.483764$ & $0.497316$ & $0.499994$ & $0$ & $1$ \\ \hline 
    
     $5$ &  $0.432121$ & $0.431269$ & $0.483764$ & $0.499952$ & $0$ & $10$ \\ \hline 
    
     $5$ &  $0.019704$ & $0.173868$ & $0.432121$ & $0.497316$ & $0$ & $1000$ \\ \hline 
    
     $6$ &  $\infty$ & $\infty$ & $\infty$ & $\infty$ & $0$ & $1$ \\ \hline 
    
     $6$ &  $\infty$ & $\infty$ & $\infty$ & $\infty$ & $0$ & $10$ \\ \hline 
    
     $6$ &  $\infty$ & $\infty$ & $\infty$ & $\infty$ & $0$ & $1000$ \\ \hline 
    \end{tabular} 
    \end{center}\caption{Результаты вычислений}  
    \label{Aggreg1CU} \end{table} 
     
     
     

    В таблице \ref{Aggreg1CU} в первом столбце указывается номер схемы;
    $E_n=\max _{x_k}\left|y\left(x_k\right)-y_k\right|, y_k$ 
    - решение соответствующей схемы при $h=10^{-n}$; $m$ - порядок сходимости, т.е. $E_n \sim O\left(h^m\right)$;
    параметр задачи $A=1,10,1000$. 
    \par
    Видно, что схемы обладающие $\alpha$-устойчивостью дают хорошее приближение при достаточно малом $h$. 
     
     


\end{document}