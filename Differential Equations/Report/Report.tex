\documentclass[14pt,a4paper]{extarticle}

\usepackage[utf8]{inputenc}
\usepackage[T2A]{fontenc}
\usepackage{amssymb,amsmath,mathrsfs,amsthm}
\usepackage[russian]{babel}
\usepackage{graphicx}
\usepackage[footnotesize]{caption2}
\usepackage{indentfirst}
\usepackage{multicol}
\usepackage{listings}
\usepackage{float}
\usepackage{url}
\usepackage{hyperref}
\usepackage{enumitem}

%\usepackage[ruled,section]{algorithm}
%\usepackage[noend]{algorithmic}
%\usepackage[all]{xy}
\usepackage{booktabs}
\usepackage{graphicx}
\usepackage[table,xcdraw]{xcolor}
\usepackage{tcolorbox}

%Библиотека для блок-схем
\usepackage{tikz}
\usetikzlibrary{shapes,arrows}

% Параметры страницы
\textheight=24cm
\textwidth=16cm
\oddsidemargin=5mm
\evensidemargin=-5mm
\marginparwidth=36pt
\topmargin=-1cm
\footnotesep=3ex
%\flushbottom
\raggedbottom
\tolerance 3000
% подавить эффект "висячих стpок"
\clubpenalty=10000
\widowpenalty=10000
%\renewcommand{\baselinestretch}{1.1}
\renewcommand{\baselinestretch}{1.5} %для печати с большим интервалом

\newcommand{\angstrom}{\mbox{\normalfont\AA}}

\newtheorem{definition}{Определение} % задаём выводимое слово (для определений)
\newtheorem{example}{Замечание} % задаём выводимое слово (для определений)
\newtheorem{theorem}{Теорема} % задаём выводимое слово (для определений)
\newtheorem{construction}{Конструкция} % задаём выводимое слово (для определений)

\DeclareMathOperator*{\sgn}{sgn}
\DeclareMathOperator*{\var}{var}
\DeclareMathOperator*{\cov}{cov}
\DeclareMathOperator*{\law}{Law}

\newcommand{\1}{\mathbbm{1}}
\newcommand{\R}{\mathbb{R}}
\newcommand{\N}{\mathbb{N}}
\newcommand{\Z}{\mathbb{Z}}
\renewcommand{\P}{\mathbb{P}}
\newcommand{\E}{\mathbb{E}}

\newcommand{\independent}{\perp\!\!\!\!\perp}

\newcommand\cA{{\cal A}}
\newcommand\cE{{\cal E}}
\newcommand\cC{{\cal C}}
\newcommand\cF{{\cal F}}
\newcommand\cG{{\cal G}}
\newcommand\cK{{\cal K}}
\newcommand\cL{{\cal L}}
\newcommand\cB{{\cal B}}
\newcommand\cN{{\cal N}}
\newcommand\cM{{\cal M}}
\newcommand\cX{{\cal X}}
\newcommand\cD{{\cal D}}
\newcommand\cR{{\cal R}}
\newcommand\cP{{\cal P}}
\newcommand\cQ{{\cal Q}}
\newcommand\cS{{\cal S}}
\newcommand\cT{{\cal T}}
\newcommand\cV{{\cal V}}
\newcommand\cZ{{\cal Z}}

\newcommand{\textProposition}    {Предложение}
\newcommand{\textTask}    {Задача}

\begin{document}

\begin{center}

    {Всеволод Заостровский, 409 группа}\\
    {\bfseries Отчёт по задаче ''Итерационные методы решения систем линейных уравнений''.\\}
    \vspace{1cm}

\end{center}

\textbf{Постановка задачи.} Для построения приближенного решения задачи
$$
y^{\prime}(x)+A y(x)=0, \quad y(0)=1, \quad x \in[0,1]
$$

с известным точным решением $y(x)=e^{-A x}$ рассматриваются следующие схемы: \\
1) $\frac{y_{k+1}-y_k}{h}+A y_k=0, y_0=1$. \\
2) $\frac{y_{k+1}-y_k}{h}+A y_{k+1}=0, y_0=1$.\\
3) $\frac{y_{k+1}-y_k}{h}+A \frac{y_{k+1}+y_k}{2}=0, y_0=1$. \\
4) $\frac{y_{k+1}-y_{k-1}}{2 h}+A y_k=0, y_0=1, y_1=1-A h$.\\
5) $\frac{1.5 y_k-2 y_{k-1}+0.5 y_{k-2}}{h}+A y_k=0, y_0=1, y_1=1-A h$.\\
6) $\frac{-0.5 y_{k+2}+2 y_{k+1}-1.5 y_k}{h}+A y_k=0, y_0=1, y_1=1-A h$.\\

Найти порядок аппроксимации, исследовать $\alpha$-устойчивость предложенных схем. Реализовать указанные схемы и заполнить таблицу.

\textbf{Решение.} 
Реализацию кода см. \href{https://github.com/VsevolodZaostrovsky/NumericalMethods/tree/main/Differential%20Equations/Code/src}{тут}. 
Для реализации численного решения необходимо в каждом случае выразить последний $y_k$ через предыдущие, заодно проверим 
$\alpha$-устойчивость и $A$-устойчивость: \\
\textbf{Схема 1.} $\frac{y_{k+1}-y_k}{h}= - A y_k $ \\
    Сходимость:\\ $y_{k+1}-y_k \text{ vs } y'(x_k) h $\\ 
    $y_{k} + h y'(x_k) + O(h^2) - y_k \text{ vs } y'(x_k) h \Rightarrow {|\frac{y_{k+1}-y_k}{h} - y'(x_k)| = O(h)}$ \\
    $A$-устойчивость:\\ $y_{k+1} = y_k (1 - Ah)$. \\
    $\lambda = 1 - A h$. \\
    $\alpha$-устойчивость:\\ $y_{k+1} - y_k = 0$. \\
    $\lambda = 1$. \\
\textbf{Вывод:} $\alpha$-устойчивость есть, $A$-устойчивость есть не всегда, $m=1$.
\par
\textbf{Схема 2.} $\frac{y_{k+1}-y_k}{h} = -A y_{k+1}$ \\ 
Сходимость:\\
$y_{k+1}-y(x_{k+1} - h) \text{ vs } y'(x_{k+1}) h $\\ 
$y_{k+1} -( y(x_{k+1}) - h y'(x_{k+1}) + O(h^2)) \text{ vs } y'(x_{k+1}) h \Rightarrow {|\frac{y_{k+1}-y_k}{h} - y'(x_{k+1})| = O(h)}$ \\ 
$A$-устойчивость:\\ 
$y_{k+1} = \frac{y_k}{1 + Ah} $.\\
$\lambda = \frac{1}{1 + Ah} < 1$. \\
$\alpha$-устойчивость:\\ $y_{k+1} - y_k = 0$. \\
$\lambda = 1$. \\
\textbf{Вывод:} $\alpha$-устойчивость есть, $A$-устойчивость есть, $m=1$.
\par
\textbf{Схема 3.} $\frac{y_{k+1}-y_k}{h} = -A \frac{y_{k+1}+y_k}{2} $\\ 
Сходимость:\\
$y(x_k + \frac{h}{2} + \frac{h}{2}) - y(x_k + \frac{h}{2} - \frac{h}{2}) \text{ vs } y'(\frac{x_{k+1} + x_k}{2}) h = y'(\frac{x_{k} + h + x_k}{2}) h$\\ 
$y(x_k + \frac{h}{2} + \frac{h}{2}) = y(x_k + \frac{h}{2}) + y'(x_k + \frac{h}{2}) \frac{h}{2} + \frac{1}{2} y''(x_k + \frac{h}{2}) \frac{h^2}{4} + O(h^3)$\\
$y(x_k - \frac{h}{2} + \frac{h}{2}) = y(x_k + \frac{h}{2}) - y'(x_k + \frac{h}{2}) \frac{h}{2} + \frac{1}{2} y''(x_k + \frac{h}{2}) \frac{h^2}{4} - O(h^3)$\\ 
$|\frac{y_{k+1}-y_k}{h} - y'(x_k + \frac{h}{2})| = O(h^2)$\\ 
$A$-устойчивость:\\ 
$y_{k+1} = \frac{y_k (2 - A h)}{2 + Ah} $. \\
$\lambda = \frac{2 - A h}{2+ A h} < 1$. \\
$\alpha$-устойчивость:\\ $y_{k+1} - y_k = 0$. \\
$\lambda = 1$. \\
\textbf{Вывод:} $\alpha$-устойчивость есть, $A$-устойчивость есть, $m=2$.
\par
\textbf{Схема 4.} $\frac{y_{k+1}-y_{k-1}}{2 h} = -A y_k$ \\
Сходимость:\\
$|\frac{y_{k+1}-y_{k-1}}{2h} - y'(x_k)| = \frac{1}{2 h} |y(x_{k} + h) - y(x_{k} - h) - 2 y'(x_k) h| =$\\
$= \frac{1}{2 h} |y(x_{k}) + y'(x_{k}) h + y''(x_{k}) \frac{h^2}{2} + y'''(x_{k}) \frac{h^3}{6} + O(h^4) \\ 
- (y(x_{k}) - y'(x_{k}) h + y''(x_{k}) \frac{h^2}{2} - y'''(x_{k}) \frac{h^3}{6} + O(h^4)) - 2 y'(x_k) h| = O(h^2)$.\\
$A$-устойчивость:\\ 
 $y_{k+2} = y_{k} - 2 A h y_{k+1} $.\\
 $\lambda^2 + 2 A h \lambda - 1 = 0$. \\
 $\lambda_{+, -} = A h \pm \sqrt{A^2 h^2 + 1}$. \\
 $|\lambda_{+}| > 1 $. \\
$\alpha$-устойчивость:\\ $y_{k+1} - y_{k-1} = 0$. \\
$\lambda = \pm 1$. \\
 \textbf{Вывод:} $\alpha$-устойчивости нет, $A$-устойчивость есть не всегда, $m=2$. 
 \par
 \textbf{Схема 5.} $\frac{1.5 y_k-2 y_{k-1}+0.5 y_{k-2}}{h} = -A y_k$ \\ 
 Сходимость:\\
$|\frac{1.5 y_k-2 y_{k-1}+0.5 y_{k-2}}{h} -y'(x_k)| = \frac{1}{h} |1.5 y_k-2(y(x_k) - y'(x_k)h +\frac{1}{2} y''(x_k)h^2 + O(h^3))+$
$0.5 ((y(x_k) - 2 y'(x_k)h +2 y''(x_k)h^2 + O(h^3))) -y'(x_k) h| = O(h^2)$\\
$A$-устойчивость:\\ 
$y_{k+2} = \frac{2 y_{k+1} - 0.5 y_{k}}{A h + 1.5} $.\\
$\lambda^2 - \frac{2}{1.5 + Ah} \lambda + \frac{0.5}{1.5 + Ah} = 0$.\\
$\lambda_{+, -} = \frac{1}{1.5 + Ah} \pm \sqrt{\frac{1}{(1.5 + Ah)^2} - \frac{1}{3 + 2 Ah} }$. \\
$\alpha$-устойчивость:\\ 
$1.5 y_k-2 y_{k-1}+0.5 y_{k-2} = 0$ \\ 
$\lambda_{\pm} = \frac{2 \pm 1}{3} = \{1, \frac{1}{3}\}$ \\ 
\textbf{Вывод:} $\alpha$-устойчивости нет, $A$-устойчивость есть не всегда, $m=2$.
\par
\textbf{Схема 6.} $\frac{-0.5 y_{k+2}+2 y_{k+1}-1.5 y_k}{h}= -A y_k$ \\ 
Сходимость:\\
$|\frac{-0.5 y_k + 2 y_{k-1} - 1.5 y_{k-2}}{h} -y'(x_k)| = \frac{1}{h} |-0.5 y_k+2(y(x_k) - y'(x_k)h +\frac{1}{2} y''(x_k)h^2 + O(h^3))$
$-1.5 ((y(x_k) - 2 y'(x_k)h +2 y''(x_k)h^2 + O(h^3))) -y'(x_k) h| = O(h)$\\
$A$-устойчивость:\\ 
$y_{k+2} = (2 A h - 3) y_k + 4 y_{k+1}$.\\
$\lambda^2  - 4 \lambda - (2 A h - 3) = 0$.\\
$\lambda_{\pm} = 2 \pm \sqrt{1 - 2 A h}$.\\
$\alpha$-устойчивость:\\ 
$-0.5 y_k+2 y_{k-1}-1.5 y_{k-2} = 0$ \\ 
$\lambda_{\pm} = \frac{2 \pm 1}{-1} = \{-3, -1\}$ \\ 
\textbf{Вывод:} $\alpha$-устойчивости нет, $m=1$.

\begin{table}[h!] 
    \begin{center} 
    \begin{tabular}{|c|c|c|c|c|c|c|} 
    \hline 
    Номер  & $E_1$ & $E_2$ & $E_3$ & $E_6$ & $m$ & $A$ \\ \hline
     $1$ &  $0.019149$ & $0.001847$ & $0.000184$ & $0.000000$ & $1$ & $1.000000$ \\ \hline 
    
     $1$ &  $0.367879$ & $0.019201$ & $0.001847$ & $0.000002$ & $1$ & $10.000000$ \\ \hline 
    
     $1$ &  $>1e5$ & $>1e5$ & $0.367879$ & $0.000184$ & $1$ & $1000.000000$ \\ \hline 
    
     $2$ &  $0.017528$ & $0.001832$ & $0.000184$ & $0.000000$ & $1$ & $1.000000$ \\ \hline 
    
     $2$ &  $0.132121$ & $0.017664$ & $0.001832$ & $0.000002$ & $1$ & $10.000000$ \\ \hline 
    
     $2$ &  $0.009901$ & $0.090864$ & $0.132121$ & $0.000184$ & $1$ & $1000.000000$ \\ \hline 
    
     $3$ &  $0.000305$ & $0.000003$ & $0.000000$ & $0.000000$ & $2$ & $1.000000$ \\ \hline 
    
     $3$ &  $0.034546$ & $0.000307$ & $0.000003$ & $0.000000$ & $2$ & $10.000000$ \\ \hline 
    
     $3$ &  $0.960784$ & $0.666712$ & $0.034546$ & $0.000000$ & $2$ & $1000.000000$ \\ \hline 
    
     $4$ &  $0.006497$ & $0.000070$ & $0.000001$ & $0.000000$ & $2$ & $1.000000$ \\ \hline 
    
     $4$ &  $408.000123$ & $48.649591$ & $0.545051$ & $0.000001$ & $2$ & $10.000000$ \\ \hline 
    
     $4$ &  $>1e5$ & $>1e5$ & $>1e5$ & $>1e5$ & $2$ & $1000.000000$ \\ \hline 
    
     $5$ &  $0.006443$ & $0.000073$ & $0.000001$ & $0.000000$ & $2$ & $1.000000$ \\ \hline 
    
     $5$ &  $0.367879$ & $0.006443$ & $0.000073$ & $0.000000$ & $2$ & $10.000000$ \\ \hline 
    
     $5$ &  $99.000000$ & $9.000045$ & $0.367879$ & $0.000001$ & $2$ & $1000.000000$ \\ \hline 
    
     $6$ &  $3388.023670$ & $>1e5$ & $>1e5$ & $>1e5$ & $1$ & $1.000000$ \\ \hline 
    
     $6$ &  $4096.000123$ & $>1e5$ & $>1e5$ & $>1e5$ & $1$ & $10.000000$ \\ \hline 
    
     $6$ &  $>1e5$ & $>1e5$ & $>1e5$ & $>1e5$ & $1$ & $1000.000000$ \\ \hline 
    \end{tabular} 
    \end{center}\caption{Результаты вычислений}  
    \label{Aggreg1CU} \end{table} 
          

    В таблице \ref{Aggreg1CU} в первом столбце указывается номер схемы;
    $E_n=\max _{x_k}\left|y\left(x_k\right)-y_k\right|, y_k$ 
    - решение соответствующей схемы при $h=10^{-n}$; $m$ - порядок сходимости, т.е. $E_n \sim O\left(h^m\right)$;
    параметр задачи $A=1,10,1000$. 
    \par
    Видно, что схемы обладающие $\alpha$-устойчивостью дают хорошее приближение при достаточно малом $h$. 
     
     


\end{document}