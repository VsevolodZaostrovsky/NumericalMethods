\documentclass[14pt,a4paper]{extarticle}

\usepackage[utf8]{inputenc}
\usepackage[T2A]{fontenc}
\usepackage{amssymb,amsmath,mathrsfs,amsthm}
\usepackage[russian]{babel}
\usepackage{graphicx}
\usepackage[footnotesize]{caption2}
\usepackage{indentfirst}
\usepackage{multicol}
\usepackage{listings}
\usepackage{float}
\usepackage{url}

\usepackage{enumitem}

%\usepackage[ruled,section]{algorithm}
%\usepackage[noend]{algorithmic}
%\usepackage[all]{xy}
\usepackage{booktabs}
\usepackage{graphicx}
\usepackage[table,xcdraw]{xcolor}
\usepackage{tcolorbox}

%Библиотека для блок-схем
\usepackage{tikz}
\usetikzlibrary{shapes,arrows}

% Параметры страницы
\textheight=24cm
\textwidth=16cm
\oddsidemargin=5mm
\evensidemargin=-5mm
\marginparwidth=36pt
\topmargin=-1cm
\footnotesep=3ex
%\flushbottom
\raggedbottom
\tolerance 3000
% подавить эффект "висячих стpок"
\clubpenalty=10000
\widowpenalty=10000
%\renewcommand{\baselinestretch}{1.1}
\renewcommand{\baselinestretch}{1.5} %для печати с большим интервалом

\newcommand{\angstrom}{\mbox{\normalfont\AA}}

\newtheorem{definition}{Определение} % задаём выводимое слово (для определений)
\newtheorem{example}{Замечание} % задаём выводимое слово (для определений)
\newtheorem{theorem}{Теорема} % задаём выводимое слово (для определений)
\newtheorem{construction}{Конструкция} % задаём выводимое слово (для определений)

\DeclareMathOperator*{\sgn}{sgn}
\DeclareMathOperator*{\var}{var}
\DeclareMathOperator*{\cov}{cov}
\DeclareMathOperator*{\law}{Law}

\newcommand{\1}{\mathbbm{1}}
\newcommand{\R}{\mathbb{R}}
\newcommand{\N}{\mathbb{N}}
\newcommand{\Z}{\mathbb{Z}}
\renewcommand{\P}{\mathbb{P}}
\newcommand{\E}{\mathbb{E}}

\newcommand{\independent}{\perp\!\!\!\!\perp}

\newcommand\cA{{\cal A}}
\newcommand\cE{{\cal E}}
\newcommand\cC{{\cal C}}
\newcommand\cF{{\cal F}}
\newcommand\cG{{\cal G}}
\newcommand\cK{{\cal K}}
\newcommand\cL{{\cal L}}
\newcommand\cB{{\cal B}}
\newcommand\cN{{\cal N}}
\newcommand\cM{{\cal M}}
\newcommand\cX{{\cal X}}
\newcommand\cD{{\cal D}}
\newcommand\cR{{\cal R}}
\newcommand\cP{{\cal P}}
\newcommand\cQ{{\cal Q}}
\newcommand\cS{{\cal S}}
\newcommand\cT{{\cal T}}
\newcommand\cV{{\cal V}}
\newcommand\cZ{{\cal Z}}

\newcommand{\textProposition}    {Предложение}
\newcommand{\textTask}    {Задача}

\begin{document}

\begin{center}

    {Всеволод Заостровский, 409 группа}\\
    {\bfseries Отчёт по задаче ''Итерационные методы решения систем линейных уравнений''.\\}
    \vspace{1cm}

\end{center}

\section{\textbf{Задача.}} Для построения приближенного решения задачи
$$
y^{\prime}(x)+A y(x)=0, \quad y(0)=1, \quad x \in[0,1]
$$

с известным точным решением $y(x)=e^{-A x}$ рассматриваются следующие схемы:
1) $\frac{y_{k+1}-y_k}{h}+A y_k=0, y_0=1$.
2) $\frac{y_{k+1}-y_k}{h}+A y_{k+1}=0, y_0=1$.
3) $\frac{y_{k+1}-y_k}{h}+A \frac{y_{k+1}+y_k}{2}=0, y_0=1$.
4) $\frac{y_{k+1}-y_{k-1}}{2 h}+A y_k=0, y_0=1, y_1=1-A h$.
5) $\frac{1.5 y_k-2 y_{k-1}+0.5 y_{k-2}}{h}+A y_k=0, y_0=1, y_1=1-A h$.
6) $\frac{-0.5 y_{k+2}+2 y_{k+1}-1.5 y_k}{h}+A y_k=0, y_0=1, y_1=1-A h$.

Найти порядок аппроксимации, исследовать $\alpha$-устойчивость предложенных схем. Реализовать указанные схемы и заполнить таблицу:
\begin{tabular}{|l|l|l|l|l|l|l|}
\hline № & $E_1$ & $E_2$ & $E_3$ & $E_6$ & $m$ & $A$ \\
\hline & & & & & & \\
\hline
\end{tabular}

Здесь
в первом столбце указывается номер схемы;
$E_n=\max _{x_k}\left|y\left(x_k\right)-y_k\right|, y_k$ 
- решение соответствующей схемы при $h=10^{-n}$; $m$ - порядок сходимости, т.е. $E_n \sim O\left(h^m\right)$;
параметр задачи $A=1,10,1000$.

\textbf{Решение.} 
Реализацию кода см. \href[здесь]{}


\end{document}