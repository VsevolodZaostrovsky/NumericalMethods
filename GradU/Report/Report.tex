\documentclass[14pt,a4paper]{extarticle}

\usepackage[utf8]{inputenc}
\usepackage[T2A]{fontenc}
\usepackage{amssymb,amsmath,mathrsfs,amsthm}
\usepackage[russian]{babel}
\usepackage{graphicx}
\usepackage[footnotesize]{caption2}
\usepackage{indentfirst}
\usepackage{multicol}
\usepackage{listings}
\usepackage{float}
\usepackage{url}
\usepackage{amsmath}

\usepackage{enumitem}

%\usepackage[ruled,section]{algorithm}
%\usepackage[noend]{algorithmic}
%\usepackage[all]{xy}
\usepackage{booktabs}
\usepackage{graphicx}
\usepackage[table,xcdraw]{xcolor}
\usepackage{tcolorbox}

%Библиотека для блок-схем
\usepackage{tikz}
\usetikzlibrary{shapes,arrows}

% Параметры страницы
\textheight=24cm
\textwidth=16cm
\oddsidemargin=5mm
\evensidemargin=-5mm
\marginparwidth=36pt
\topmargin=-1cm
\footnotesep=3ex
%\flushbottom
\raggedbottom
\tolerance 3000
% подавить эффект "висячих стpок"
\clubpenalty=10000
\widowpenalty=10000
%\renewcommand{\baselinestretch}{1.1}
\renewcommand{\baselinestretch}{1.5} %для печати с большим интервалом

\newcommand{\angstrom}{\mbox{\normalfont\AA}}

\newtheorem{definition}{Определение} % задаём выводимое слово (для определений)
\newtheorem{example}{Замечание} % задаём выводимое слово (для определений)
\newtheorem{theorem}{Теорема} % задаём выводимое слово (для определений)
\newtheorem{proposition}{Утверждение} % задаём выводимое слово (для определений)
\newtheorem{construction}{Конструкция} % задаём выводимое слово (для определений)

\DeclareMathOperator*{\sgn}{sgn}
\DeclareMathOperator*{\var}{var}
\DeclareMathOperator*{\cov}{cov}
\DeclareMathOperator*{\law}{Law}

\newcommand{\1}{\mathbbm{1}}
\newcommand{\R}{\mathbb{R}}
\newcommand{\N}{\mathbb{N}}
\newcommand{\Z}{\mathbb{Z}}
\renewcommand{\P}{\mathbb{P}}
\newcommand{\E}{\mathbb{E}}

\newcommand{\independent}{\perp\!\!\!\!\perp}

\newcommand\cA{{\cal A}}
\newcommand\cE{{\cal E}}
\newcommand\cC{{\cal C}}
\newcommand\cF{{\cal F}}
\newcommand\cG{{\cal G}}
\newcommand\cK{{\cal K}}
\newcommand\cL{{\cal L}}
\newcommand\cB{{\cal B}}
\newcommand\cN{{\cal N}}
\newcommand\cM{{\cal M}}
\newcommand\cX{{\cal X}}
\newcommand\cD{{\cal D}}
\newcommand\cR{{\cal R}}
\newcommand\cP{{\cal P}}
\newcommand\cQ{{\cal Q}}
\newcommand\cS{{\cal S}}
\newcommand\cT{{\cal T}}
\newcommand\cV{{\cal V}}
\newcommand\cZ{{\cal Z}}

\newcommand{\textProposition}    {Предложение}
\newcommand{\textTask}    {Задача}

\begin{document}

\begin{center}
    {Всеволод Заостровский, 409 группа}\\
    {\bfseries Отчёт по задаче ''Решение страшного уравнения''.\\}
    \vspace{1cm}
\end{center}

\tableofcontents

\section{Постановка задачи.} \label{diffeq1}
\subsection{Одномерный Лаплас}
Необходимо решить уравнение:
\begin{equation*} 
    u_t(t, x) = \operatorname{div} (k(x) \operatorname{grad} u(t, x)).
\end{equation*}
Будем считать, что $0 \leq t,x \leq 1$. В моём варианте, краевые условия:
\begin{align*} 
    &u(t, x) \big| _{x \in \partial \Omega} = 0, \quad \Omega = [0,1]. \\
    &u(0, x) = u^0(x), \quad x \in \Omega. 
\end{align*}

\subsection{Двумерный Лаплас}
Необходимо решить уравнение:
\begin{equation*} 
    u_t(t, x, y) = \operatorname{div} (k(x, y) \operatorname{grad} u(t, x, y)).
\end{equation*}
Будем считать, что $0 \leq t,x,y \leq 1$. В моём варианте, краевые условия:
\begin{align*} 
    &u(t, x, y) \big| _{(x ,y) \in \partial \Omega} = 0, \quad \Omega = [0,1] \times [0,1]. \\
    &u(0, x, y) = u^0(x, y), \quad (x, y) \in \Omega. 
\end{align*}

\section{Алгоритм решения одномерной схемы.}
\subsection{Дискретизация}
Уравнение будем приближать посредством следующей схемы:
\begin{align*}
    &\frac{u^{n+1}_i - u^n_i}{\tau} = \frac{k(x_{i + \frac{1}{2}}) \frac{u^{n+1}_{i+1} - u^{n+1}_{i}}{h} - k(x_{i- \frac{1}{2}}) \frac{u^{n+1}_{i} - u^{n+1}_{i-1}}{h}}{h}.
\end{align*}
Краевые условия:
\begin{align*} 
    &u(t, x) \big| _{x \in \partial \Omega} = 0, \quad \Omega = [0,1]. \\
    &u(0, x) = u^0(x), \quad x \in \Omega. 
\end{align*}
\subsection{Общий вид матрицы уравнения.}
Приведем схему к матричному виду:
\begin{align*}
    u^{n+1}_i - u^n_i = \tau k(x_{i + \frac{1}{2}}) \frac{u^{n+1}_{i+1} - u^{n+1}_{i}}{h^2} - \tau k(x_{i- \frac{1}{2}}) \frac{u^{n+1}_{i} - u^{n+1}_{i-1}}{h^2}.\\
    u^{n+1}_i - u^n_i = 
    \tau k(x_{i + \frac{1}{2}}) \frac{u^{n+1}_{i+1} }{h^2} 
    - u^{n+1}_i \frac{\tau}{h^2} (k(x_{i + \frac{1}{2}}) + k(x_{i - \frac{1}{2}})) 
    + \tau k(x_{i- \frac{1}{2}}) \frac{u^{n+1}_{i-1}}{h^2}. 
\end{align*}
Таким образом, получим:
\begin{equation}
    - u^{n+1}_{i+1} k(x_{i + \frac{1}{2}}) \frac{\tau}{h^2} +
    u^{n+1}_i\left(1 + \frac{\tau}{h^2} (k(x_{i + \frac{1}{2}}) + k(x_{i - \frac{1}{2}}))\right) - u^{n+1}_{i-1} k(x_{i- \frac{1}{2}}) \frac{ \tau}{h^2} = u^n_i.
\end{equation}
Матрица примет вид: \\
$$A = \begin{pmatrix}
    C   & B_- & 0     & 0   & 0   & \ldots & 0 \\
    B_+ & C   & B_-   & 0   & 0   & \ldots & 0 \\
    0   & B_+ & C     & B_- & 0   & \ldots & 0 \\
    0   & 0   & B_+   & C   & B_- & \ldots & 0 \\
    \ldots & \ldots & \ldots & \ldots & \ldots & \ldots & \ldots \\
    0   & 0   & 0    & \ldots & B_+ & C & B_- \\
    0   & 0   & 0     & \ldots & 0 & B_+ & C \\
\end{pmatrix}, \text{где}
\left\{\begin{array}{l}
 C = 1 + \frac{\tau}{h^2} (k(x_{i + \frac{1}{2}}) + k(x_{i - \frac{1}{2}})), \\
 B_+ = k(x_{i + \frac{1}{2}}) \frac{\tau}{h^2}, \\
 B_- = k(x_{i - \frac{1}{2}}) \frac{\tau}{h^2}.
\end{array}\right. 
$$
\subsection{Решение.}
Методом прогонки...

\section{Алгоритм решения двумерной схемы.}
\subsection{Дискретизация.}
\begin{align*}
    \frac{u^{n+1}_i - u^n_i}{\tau} &= \frac{k(x_{i + \frac{1}{2}, j}) \frac{u^{n+1}_{i+1, j} - u^{n+1}_{i, j}}{h} - k(x_{i- \frac{1}{2}, j}) \frac{u^{n+1}_{i, j} - u^{n+1}_{i-1, j}}{h}}{h} + \\
    &+  \frac{k(x_{i, j + \frac{1}{2}}) \frac{u^{n+1}_{i, j+1} - u^{n+1}_{i, j}}{h} - k(x_{i, j- \frac{1}{2}}) \frac{u^{n+1}_{i, j} - u^{n+1}_{i, j-1}}{h}}{h}.
\end{align*}

 \end{document}