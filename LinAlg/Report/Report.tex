\documentclass[14pt,a4paper]{extarticle}

\usepackage[utf8]{inputenc}
\usepackage[T2A]{fontenc}
\usepackage{amssymb,amsmath,mathrsfs,amsthm}
\usepackage[russian]{babel}
\usepackage{graphicx}
\usepackage[footnotesize]{caption2}
\usepackage{indentfirst}
\usepackage{multicol}
\usepackage{listings}
\usepackage{float}
\usepackage{url}

\usepackage{enumitem}

%\usepackage[ruled,section]{algorithm}
%\usepackage[noend]{algorithmic}
%\usepackage[all]{xy}
\usepackage{booktabs}
\usepackage{graphicx}
\usepackage[table,xcdraw]{xcolor}
\usepackage{tcolorbox}

%Библиотека для блок-схем
\usepackage{tikz}
\usetikzlibrary{shapes,arrows}

% Параметры страницы
\textheight=24cm
\textwidth=16cm
\oddsidemargin=5mm
\evensidemargin=-5mm
\marginparwidth=36pt
\topmargin=-1cm
\footnotesep=3ex
%\flushbottom
\raggedbottom
\tolerance 3000
% подавить эффект "висячих стpок"
\clubpenalty=10000
\widowpenalty=10000
%\renewcommand{\baselinestretch}{1.1}
\renewcommand{\baselinestretch}{1.5} %для печати с большим интервалом

\newcommand{\angstrom}{\mbox{\normalfont\AA}}

\newtheorem{definition}{Определение} % задаём выводимое слово (для определений)
\newtheorem{example}{Замечание} % задаём выводимое слово (для определений)
\newtheorem{theorem}{Теорема} % задаём выводимое слово (для определений)
\newtheorem{construction}{Конструкция} % задаём выводимое слово (для определений)

\DeclareMathOperator*{\sgn}{sgn}
\DeclareMathOperator*{\var}{var}
\DeclareMathOperator*{\cov}{cov}
\DeclareMathOperator*{\law}{Law}

\newcommand{\1}{\mathbbm{1}}
\newcommand{\R}{\mathbb{R}}
\newcommand{\N}{\mathbb{N}}
\newcommand{\Z}{\mathbb{Z}}
\renewcommand{\P}{\mathbb{P}}
\newcommand{\E}{\mathbb{E}}

\newcommand{\independent}{\perp\!\!\!\!\perp}

\newcommand\cA{{\cal A}}
\newcommand\cE{{\cal E}}
\newcommand\cC{{\cal C}}
\newcommand\cF{{\cal F}}
\newcommand\cG{{\cal G}}
\newcommand\cK{{\cal K}}
\newcommand\cL{{\cal L}}
\newcommand\cB{{\cal B}}
\newcommand\cN{{\cal N}}
\newcommand\cM{{\cal M}}
\newcommand\cX{{\cal X}}
\newcommand\cD{{\cal D}}
\newcommand\cR{{\cal R}}
\newcommand\cP{{\cal P}}
\newcommand\cQ{{\cal Q}}
\newcommand\cS{{\cal S}}
\newcommand\cT{{\cal T}}
\newcommand\cV{{\cal V}}
\newcommand\cZ{{\cal Z}}

\newcommand{\textProposition}    {Предложение}
\newcommand{\textTask}    {Задача}

\begin{document}

\begin{center}

    {Всеволод Заостровский, 409 группа}\\
    {\bfseries Отчёт по задаче ''Итерационные методы решения систем линейных уравнений''.\\}
    \vspace{1cm}

\end{center}

\section{\textbf{Задача 1.}} Для решения системы линейных уравнений:
\begin{align*}
& -\frac{y_{k+1} - 2 y_k + y_{k-1}}{h^2} + p y_k = f_k, \;\;\;\;\; k = 1, \ldots, N - 1; \\
& y_0 = y_N = 0; \\
& h = \frac{1}{N}; \\
& p \geq 0.
\end{align*}
реализуйте метод Фурье (т.е. метод разложения по собственным векторам) для базисных функций:
\begin{equation}
    \psi_k^{(n)} = \sin(\frac{\pi n k}{N})
\end{equation}    

\textbf{Решение.} 

Вычислим $\lambda_n$: \\
$
{A \psi^{(n)}}_k =  -\frac{\psi^{(n)}_{k+1} - 2 \psi^{(n)}_k + \psi^{(n)}_{k-1}}{h^2} + p \psi^{(n)}_k \\
= -\frac{\sin(\frac{\pi n (k+1)}{N}) - 2 \sin(\frac{\pi n k}{N}) + \sin(\frac{\pi n (k-1)}{N})}{h^2} + p \sin(\frac{\pi n k}{N}) \\
= -\frac{\sin(\frac{\pi n k}{N}) \cos(\frac{\pi n}{N}) + \cos(\frac{\pi n k}{N}) \sin(\frac{\pi n}{N}) - 2 \sin(\frac{\pi n k}{N}) + \sin(\frac{\pi n k}{N}) \cos(\frac{\pi n}{N}) - \cos(\frac{\pi n k}{N}) \sin(\frac{\pi n}{N}) }{h^2} + p \sin(\frac{\pi n k}{N}) \\
= -\frac{\sin(\frac{\pi n k}{N}) \cos(\frac{\pi n}{N}) - 2 \sin(\frac{\pi n k}{N}) + \sin(\frac{\pi n k}{N}) \cos(\frac{\pi n}{N})}{h^2} + p \sin(\frac{\pi n k}{N}) \\
= -\frac{2 \sin(\frac{\pi n k}{N}) (\cos(\frac{\pi n}{N}) - 1)}{h^2} + p \sin(\frac{\pi n k}{N}) 
=  \sin(\frac{\pi n k}{N}) (-\frac{2 (\cos(\frac{\pi n}{N}) - 1)}{h^2} + p).
$ \\
Отсюда:
\begin{equation}
    \lambda_n = N^2 (2 (\cos(\frac{\pi n}{N}) - 1)) + p.
\end{equation}
Кроме того, из указания к задаче:
\begin{equation}
    c_n = \frac{\left(f, \psi^{(n)} \right)}{\lambda_n \left(\psi^{(n)}, \psi^{(n)} \right)} 
            = \frac{2 \left(f, \psi^{(n)} \right)}{\lambda_n}.
\end{equation}
Решение же можно найти в виде:
\begin{equation}
    y_k = \sum_{n=1}^{N-1} c_n \psi_k^{(m)}.
\end{equation}

\end{document}